\utilchap{wind2fds - A utility for converting wind data for use by FDS and Smokeview}
The utility {\tt wind2fds}\ is designed to enable Smokeview to visualize wind velocity measurements made by SODARs
(SOnic Detection And Ranging).
A SODAR uses sound waves to measure wind velocity at various heights above ground level.
Vertical wind profiles can then be visualized using Smokeview and compared with velocity profiles generated by FDS.

A typical (abbreviated) output generated by a SODAR is listed in Figure \ref{fig:winddata}\ where
the {\tt ws30}, {\tt ws35}\ headings indicate wind speed
and the {\tt wd30}, {\tt wd35}\ headings indicate wind direction.
The utility, {\tt wind2fds}, then converts this data into a form given in Figure \ref{fig:convertedwinddata}.
The converted wind data file begins with a {\tt HEADER}\ section
containing information about where various wind measurements were taken. Smokeview uses the information in the {\tt HEADER}\ section to associate wind data measurements with locations. The {\tt HEADER}\ section is followed by a {\tt DATA}\ section.  This section is identical
to the file format used by Smokeview to visualize device data.

\begin{figure}[bph]
{\small
\begin{verbatim}
date,          time, ws30, ws35, wd30, wd35
10/26/2011, 0:00:00,  6.5, 6.56,  333,  334
10/26/2011, 0:05:00, 7.83, 8.09,  336,  336
10/26/2011, 0:10:00, 8.36, 8.41,  334,  336
10/26/2011, 0:15:00, 7.6,  8.57,  335,  336
10/26/2011, 0:20:00, 7.44,  8.3,  340,  342
10/26/2011, 0:25:00, 7.92,  9.2,  343,  339
10/26/2011, 0:30:00, 9.36, 9.25,  336,  340
\end{verbatim}
}
\caption
[Experimental wind data]
{Experimental wind data.}
\label{fig:winddata}%
\end{figure}


\begin{figure}[bph]
{\small
\begin{verbatim}
//HEADER
DEVICE
 ws30 % VELOCITY % sensor
 0.0 0.0 30.0
DEVICE
 ws35 % VELOCITY % sensor
 0.0 0.0 35.0
DEVICE
 wd30 % ANGLE % sensor
 0.0 0.0 30.0
DEVICE
 wd35 % ANGLE % sensor
 0.0 0.0 35.0
//DATA
s,s,m/s,m/s,deg,deg
time,time_orig,ws30,ws35,wd30,wd35
0,0:00:00,6.5,6.56,333,334
300,0:05:00,7.83,8.09,336,336
600,0:10:00,8.36,8.41,334,336
900,0:15:00,7.6,8.57,335,336
1200,0:20:00,7.44,8.3,340,342
1500,0:25:00,7.92,9.2,343,339
1800,0:30:00,9.36,9.25,336,340
\end{verbatim}
}
\caption
[Experimental wind data converted by wind2fds for use by Smokeview]
{Experimental wind data converted by wind2fds for use by Smokeview.}
\label{fig:convertedwinddata}%
\end{figure}

\begin{figure}[bph]
{\small
\lstinputlisting{SCRIPT_FIGURES/wind2fds.help}
}
\caption
[Usage information for the program {\tt wind2fds}]
{Usage information for the program {\tt wind2fds}.}
\label{figwind2fdsusage}%
\end{figure}

{\tt wind2fds}\ allows one to link measurement locations with wind data, to exclude data before and/or after specified date/times and to label measurements as presented in Smokeview.
Help information may be obtained at the command line
by typing {\tt wind2fds -h}\ as given in Figure \ref{figwind2fdsusage}.

\begin{figure}[bph]
\begin{center}
\begin{tabular}{c}
 \includegraphics[height=5.0in]{../SMV_Verification_Guide/SCRIPT_FIGURES/wind_test1_002}
 \end{tabular}
\end{center}
\caption
[Visualization of wind data converted for use by Smokeview using wind2fds]
{Visualization of wind data converted for use by Smokeview using wind2fds. The line segments represent
 wind speed and direction.  The spherical shells represent uncertainty
 in wind direction (shell diameter) and wind speed (shell thickness).}
\label{figwind}%
\end{figure}

Figure \ref{figwind}\ gives an example of a visualization of wind data converted for use by Smokeview using wind2fds. The line segments represent
 wind speed and direction.  The spherical shells represent uncertainty
 in wind direction (shell diameter) and wind speed (shell thickness).
