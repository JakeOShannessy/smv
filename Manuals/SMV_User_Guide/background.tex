\utilchap{background - A utility for running multiple programs simultaneously}

This \textchap\ explains how to use the utility {\em background} ({\em background.exe}\ on the PC) what it is and
how it might be useful to FDS users.  It is included with the FDS/Smokeview
bundle.  A Windows user can use the {\em Start}\ command to run multiple programs at the same time.  
This is fine for a few programs but a computer can easily become
overloaded if many jobs are run using this method.  
Similarly, a Linux/Mac user can run a program appending an {\tt \&}\ character to the end of a 
command line to put a job in the background.  As on the PC, if many jobs are run, the system can become overloaded.  
The program {\em background}\ throttles job submissions so that a job won't start until the CPU load is below a
specified level.  A job submission can also be delayed until the memory usage is below a specified level.  
This enables one to submit a long list of FDS cases without saturating the CPU,
since only a small number  of jobs will be
running at any one time.

The utility {\em background.exe}\ allows parallel
processing to occur at the program level.  It is often the case that one is doing a
parameter study or running a long list of cases to verify the use of FDS. Typically you
would create a windows batch file (.bat) containing a list of commands like

\begin{lstlisting}
fds casename_1.fds
....
fds casename_n.fds
\end{lstlisting}

On a Windows system, each entry in the above list will not start until the previous entry
has completed, even if the computer has multiple cores or CPUs.
Unix/Linux based systems have the capability of putting computer jobs in the background,
meaning that when a job is run, control returns immediately allowing the next job in the
list to start running.  With computers that have multiple cores or CPUS, one can then run
more than one job simultaneously.

Here is how one might use {\em background}\ with FDS

\begin{lstlisting}
background -d 1.0 -u 90 fds casename.fds
\end{lstlisting}

This command runs ``fds casename.fds'' after waiting 1~s and ensuring that the CPU usage is
less than 90 \%. If the CPU usage happens to be more than 90 \%, the program {\em background}\
waits to submit the fds command until the usage drops below 90 \% .  Once this occurs, it runs
the command, {\tt fds casename.fds}.

The purpose of the delay before submitting a job is to give Windows a
chance to update the usage level from 
previous invocations.  This ensures that a
large number of jobs are not
submitted at once.

The background utility is designed to be used in a Windows batch file.
For example, suppose you have
a list of five FDS jobs you want to run in a Windows batch file. On a
Windows computer you would have \
a batch file containing something like

\begin{lstlisting}
fds case1.fds
fds case2.fds
fds case3.fds
fds case4.fds
fds case5.fds
\end{lstlisting}

Using background with a 2 second delay and 75 per cent maximum
load level, you would change your script to something like

\begin{lstlisting}
background -d 2 -u 75 fds case1.fds
background -d 2 -u 75 fds case2.fds
background -d 2 -u 75 fds case3.fds
background -d 2 -u 75 fds case4.fds
background -d 2 -u 75 fds case5.fds
\end{lstlisting}

Usage information for {\tt background}\ may be obtained
by typing {\tt background -h}\ which gives output listed in Figure \ref{fig:background}.

\begin{figure}[bph]
{\small
\lstinputlisting{SCRIPT_FIGURES/background.help}
}
\caption
[Usage information for the program {\tt background}]
{Usage information for the program {\tt background}}
\label{fig:background}%
\end{figure}
