\documentclass[11pt]{article}

\usepackage{graphicx}
\usepackage{pdfsync}
\usepackage[colorlinks=true,pagebackref]{hyperref}
\usepackage{stfloats}
\usepackage{lipsum}
\usepackage{authblk}
\usepackage{fancyhdr}

\renewcommand\refname{\vskip-2.5\baselineskip\normalsize\underline{\bf REFERENCES}}

\setlength{\textwidth}{6.5in}
\setlength{\textheight}{9.0in}
\setlength{\topmargin}{-.5in}
\setlength{\headheight}{18pt}
\setlength{\headsep}{0.in}
\setlength{\parindent}{0.25in}
\setlength{\oddsidemargin}{0.0in}
\setlength{\evensidemargin}{0.0in}
\setlength{\parindent}{0.0in}
\setlength{\parskip}{\baselineskip}
\setlength{\columnsep}{0.5in}

\usepackage{unicode-math}
\usepackage{fontspec}
\setmainfont{Cambria}
\setmathfont{Cambria Math}

\usepackage{siunitx}
\sisetup{
    detect-all = true,
    input-decimal-markers = {.},
    input-ignore = {,},
    inter-unit-product = \ensuremath{{}\cdot{}},
    multi-part-units = repeat,
    number-unit-product = \text{~},
    per-mode = fraction,
    separate-uncertainty = true,
}

\newcommand{\paper}{note}
\newcommand{\SMVfigdir}{../../../fig/smv/figures}
\newcommand{\vvec}[1]{\mathbf{#1}}

\title{\huge VISUALIZING SMOKE AND FIRE}

\author[1]{Glenn Forney}

\affil[1]{
National Institute of Standards and Technology\protect\\
Gaithersburg, Maryland, USA\protect\\
e-mail: glenn.forney@nist.gov}

\date{}

\renewcommand\Authfont{\small}
\renewcommand\Affilfont{\small}

\pagestyle{plain}
\fancyhf{}
\lhead{\scriptsize PROCEEDINGS, Fire and Evacuation Modeling Technical Conference 2018\\
Gaithersburg, Maryland, October 1-3, 2018}

\renewcommand{\headrulewidth}{0pt}

\newcommand{\ssection}[1]{\underline{\bf #1}}
\newcommand{\ssubsection}[1]{\underline{\bf #1}}


\begin{document}

\maketitle

\thispagestyle{fancy}

\vskip-\baselineskip
\ssection{ABSTRACT}

This \paper\ discusses some of the physics and associated numerical algorithms used by Smokeview to
visualize smoke and fire. Realistic visualization methods are important for applications where
one wishes to observe qualitative effects of fire and smoke rather than quantitative
characteristics such as temperature or velocity.
Approximations and simplifications are required to display smoke and fire at interactive frame rates.
The primary approximation is to take advantage of the low albedo character of
smoke allowing one to either simplify or eliminate scattering terms in the radiative
transport equation used to visualize smoke and fire.

% -------------------  Introduction ------------------------

\ssection{INTRODUCTION}

This \paper\ discusses some of the physics and associated numerical
algorithms used by Smokeview~\cite{Smokeview_Users_Guide,Smokeview_Tech_Guide} to
visualize smoke and fire.  Smoke color and opacity are visualized
using quantitative physics-based methods.  Flame color
is visualized using an arbitrary user-specified color palette
where color is mapped to gas temperature.
Smoke opacity is visualized using Beer's law relating smoke density and opacity.

Realistic visualization of fire is important
for applications where one wishes to observe qualitative effects
of fire and smoke rather than determine quantitative
data such as temperature or velocity.  This
would be the case for a fire fighter using a computer based fire
fighting simulator. Realistic visualization methods, however,
complement but do not replace other more traditional visualization
methods such as 2D contouring or 3D iso-surfacing which are better
suited for quantitatively analyzing data.

Complete methods for visualizing smoke and fire data taking into
account interactions between light and smoke require the solution
of the radiation transport equation (RTE)~\cite{Siegel:2001} also
called the volume rendering equation in the visualization
literature.~\cite{levoy:1988} This equation models how light is
affected after interacting with smoke, a participating medium. In
particular, Smokeview uses the RTE to account for extinction
(absorption plus out-scattering) by the smoke and emission from
the fire.

The form of the RTE used by Smokeview to model smoke and fire
appearance is identical to that used by the Fire Dynamics Simulator (FDS) to model radiative
heat transfer. Smokeview uses an extinction coefficient
appropriate for visible light while FDS use one appropriate for
infrared wavelengths of light.  With the proper extinction
coefficient, however, Smokeview can also view smoke at other
wavelengths, simulating a thermal imager, for example. Smokeview
solves the RTE assuming a gray gas environment. This is the
default solution method for FDS. One other important difference is
that Smokeview requires a solution at only one point at a time
(any arbitrary point though), the observer's viewpoint, while FDS
requires a radiation field, a solution at all points within the
solution domain.
Approximations are
required in order to display smoke and fire at interactive frame
rates.  The primary approximation is to take advantage of the low
albedo character of smoke allowing one to either simplify or
eliminate scattering terms in the RTE.

A slice rendering method for solving the RTE,  splits the
integration path at grid planes within a 3D mesh.
A series of partially transparent slices are drawn through the data where each
slice is approximately perpendicular to the line of sight.
These partially transparent slice planes are then drawn
individually and combined by the video hardware to form one image.
The spacing of data within one plane and the spacing between planes are parameters
that can be specified by the user in order to speed up the visualization.

As the separation distance between slice planes becomes smaller, the
computed opacity values are subject to increased round off error
due to finite precision arithmetic.  In fact, if these planes are
sufficiently close, the computed opacities truncate to zero.   In
this situation other techniques for visualizing smoke such as volume rendering methods are required.

% -------------------  Radiation Transport Equation ------------------------

\ssection{RADIATION TRANSPORT EQUATION}

The model used here to visualize smoke is the radiation transport
equation (RTE)~\cite{Siegel:2001}.  This equation uses radiance to
represent smoke appearance.  Radiance has units of Watts per
square meter per unit solid angle~\si{W/(sr.m^2)}.  The solid
angle accounts for the fact that a light source appears brighter
if it emits a given amount of light through a smaller
cross-sectional area.  Similarly the radiance of a light source does not
depend on distance from the
observer (unless a participating medium is present to absorb or
scatter  light) since any increase in distance that would reduce radiance is offset
by the light source's reduced cross sectional area.
The radiation transport equation discussed in
this section models the change in radiance due to these factors.

\newcommand{\dx}[1]{\,\mbox{d}#1}
\newcommand{\siga}{ \sigma_a(x) }
\newcommand{\sigt}{ \sigma_t(x) }
\newcommand{\sigs}{ \sigma_s(x) }
\newcommand{\sigts}{ \sigma_t(s) }
\newcommand{\Le}{ C_e(x) }
\newcommand{\Lexo}{ C_e(x,\omega) }
\newcommand{\Lxo}{ C(x,\omega) }
\newcommand{\dLdx}{ \frac{\dx{C}}{\dx{x}}(x)}
\newcommand{\intf}[2]{ \exp\left({\int_{#1}^{#2} \sigts \dx{s}}\right) }
\newcommand{\intff}[2]{ {\int_{#1}^{#2} \sigts \dx{s}} }
\newcommand{\intmf}[2]{ \exp\left({-\int_{#1}^{#2} \sigts \dx{s}}\right) }
\newcommand{\intmff}[2]{ {-\int_#1^#2 \sigts \dx{s}} }
\newcommand{\ddx}{ \frac{\mbox{d}}{\dx{x}} }

The radiation transport equation is used to calculate radiance due
to one or more light sources within a region possibly containing a
participating medium such as smoke~\cite{Siegel:2001}. The change
in radiance along a ray with direction $\omega$ at any one instant
and wavelength may be expressed using

\begin{eqnarray}
\label{eq:fullrte}
 \left(\omega\cdot\nabla\right)\Lxo =
-\underbrace{\siga\Lxo}_\mathrm{absorption}-\underbrace{\sigs\Lxo}_\mathrm{out-scattering}
+ \underbrace{\siga\Lexo}_\mathrm{emission} +
\underbrace{\sigs\int_{4\pi}p(x,\omega,\omega')C_i(x,\omega')\dx{\omega'}}_\mathrm{in-scattering}
\end{eqnarray}

\noindent where  $\Lxo$ represents the  radiance at $x$ along a
direction $\omega$. As illustrated in Figure~\ref{figRadiance}, the
right hand side of equation (\ref{eq:fullrte}) is split into four
components accounting for absorption, in and out scattering and
emission where $\siga$ is the absorption coefficient, $\sigs$ is
the scattering coefficient, $\Lexo$ is the radiance emitted at $x$
along a direction $\omega$ and $p(x,\omega,\omega')$ is the
fraction of light moving along direction $\omega'$ scattered along
direction $\omega$. Absorption and out-scattering cause radiance
to decrease while emission and in-scattering cause radiance to
increase. The radiance terms $C$, $C_e$ and $C_i$ have units of
\si{W/(m^2.sr)}. The coefficients $\sigma_a$ and $\sigma_s$ have
units of \si{1/m} and specify the time and location dependent
change per unit length to the radiance term to which they are
applied.

\begin{figure}[bph]
\begin{center}
\includegraphics[width=6.0in]{\SMVfigdir/rte_setup}
\end{center}
\caption{Diagram illustrating components of the
radiation transport equation.  Absorption and out-scattering terms
decrease radiance.  Emission and in-scattering terms increase
radiance} \label{figRadiance}
\end{figure}

% ----  Approximating the Radiation Transport Equation ------------------------

\ssubsection{Approximating the Radiation Transport Equation}

The RTE may be simplified by ignoring the in-scattering term and combining out-scattering
and absorption coefficients.  The Beer-Lambert law results if the emission term is also
neglected.

Equation (\ref{eq:fullrte}) is then approximated by neglecting the
integral term and using $\sigt=\siga+\sigs$ to obtain

\begin{eqnarray}
\dLdx&=&-\sigt C(x) + \siga C_e(x)\\
 C(x_0)&=&C_0
\end{eqnarray}

\noindent This equation has solution

\begin{equation}
\label{eq:rtesoln}
 C(x_N)=\tau(x_0,x_N)C_0 + \int_{x_0}^{x_N}\tau(x,x_N)\siga\Le \dx{x}
\end{equation}

\noindent where $\tau(a,b)$ representing the optical depth between $a$ and $b$ is given by
\begin{equation}
\label{eq:optdepth}
\tau(a,b)=\intmf{a}{b}
\end{equation}

\noindent If the emission term is neglected, $\sigma_t(x)=\sigma_t$ is assumed to be constant
and $L$ is the path length then this equation simplifies to
\begin{eqnarray}
 \frac{C(x_N)}{C_0}=\exp(-\sigma_tL)
\end{eqnarray}
which is the Beer-Lambert law.

% -------------------  Discretizing the Radiation Transport Equation ------------------------

\ssubsection{Discretizing the Radiation Transport Equation}

\newcommand{\htau}[1]{\tau_{#1}^{N-1}}
\newcommand{\halpha}[1]{\alpha_{#1}^{N-1}}
\newcommand{\sigai}[1]{\sigma_{a,#1}}
\newcommand{\Lei}[1]{C_{e,#1}}
\newcommand{\Lhatj}[1]{C_{#1}^N}
\newcommand{\Lhatjj}[1]{\hat{C}_{#1}^N}
\newcommand{\Chatjj}[1]{\hat{C}_{#1}^N}
\newcommand{\Leii}[1]{\hat{C}_{e,#1}}

The approximate RTE solution given in equation (\ref{eq:rtesoln}) is
discretized by converting integrals into a sum. Figure~\ref{fig:smokediscretesetup}\ illustrates the terms used to
perform these discretizations.  The integration path is split into $N$ parts
each with length $\Delta x=(x_N-x_0)/N$.  The coordinate system is
set up so that the initial radiance, $C_0$, is located at $x_0$,
most distant from the observer and the final radiance, $C_N$, is
located at $x_N$ closest to the observer.

\begin{figure}[bph]
\begin{center}
\includegraphics[width=5.0in]{\SMVfigdir/smoke_discrete_setup}
\end{center}
\caption{Setup for discretizing
the equations used to model radiance within a column of 3D smoke
data. The transparency across the interval from $x_i$ to $x_{i+1}$
is $\tau_i$. The transparency across the intervals from $x_i$ to
the observer is the product of individual transparencies or
$\tau_i\tau_{i+1}\cdots\tau_{N-1}$}
\label{fig:smokediscretesetup}
\end{figure}

The optical depth, $\tau(a,b)$, defined in equation (\ref{eq:optdepth}) is
discretized using a sum  after defining sample points
$s_j=x_0+j\Delta s$ for $j=0$ to $N$ with spacing $\Delta
s=(x_N-x_0)/N$ to obtain

\begin{eqnarray}
\htau{i}=\tau(x_i,x_N)&=&\exp\left(-\int_{x_i}^{x_N}\sigma_t(s)\dx{s}\right)\approx\exp\left(-\sum_{j=i}^{N-1}\sigma_t(s_j)\Delta s\right)\\
&=&\prod_{j=i}^{N-1}\exp\left(-\sigma_t(s_j)\Delta s\right)=\prod_{j=i}^{N-1}\tau_j
\end{eqnarray}

\noindent where $\tau_j=\exp\left(-\sigma_t(s_j)\Delta s\right)$
represents the transparency over one discretization interval.
For~$i=N-1$~to~$0$, the optical depth $\htau{i}$ may be computed
recursively using
\begin{eqnarray}
\label{eq:tauhat_recurse}
\htau{i}&=&\htau{i+1}\tau_i
\end{eqnarray}
\noindent where the recursion is initiated with $\htau{N}=1$.
Substituting $1-\halpha{i}=\htau{i}$ and $1-\alpha_i=\tau_i$ into
equation (\ref{eq:tauhat_recurse}) gives
\begin{eqnarray}
1-\halpha{i}=(1-\halpha{i+1})(1-\alpha_i)=1-\halpha{i+1} - \alpha_i + \halpha{i+1}\alpha_i
\end{eqnarray}
which simplifies to
\begin{eqnarray}
\label{eq:alpha2}
\halpha{i}&=&\halpha{i+1} + (1-\halpha{i+1})\alpha_i
\end{eqnarray}

Similarly, the radiance given by the RTE solution $C(x_N)$ in
equation (\ref{eq:rtesoln}) may be discretized to obtain

\begin{equation}
C_{N} = \htau{0}\,C_0 +
\sum_{i=0}^{N-1}\htau{i}\,\sigai{i}\,\Lei{i}\,\Delta x
\label{eq:discrete_rte2}
\end{equation}

\noindent What is seen on the screen is the term $C_N$.
If the expression $\sigma_{a,i}C_{e,i}\Delta x$ in equation (\ref{eq:discrete_rte2}) is interpreted as
the emitted color of the fire or
heated gas at a location $i$ and $C_0$ is interpreted as the color
of the light source {\em behind}\ the smoke, then equation
(\ref{eq:discrete_rte2}) restated in words gives the observed color
as a weighted average of source and emitted
colors where each weight is the optical depth from the observer to
the corresponding emitted color location.  These emitted colors can be
determined from a blackbody temperature curve or from a colormap
meant to show variations in temperature in terms of color.

% -------------------  A Solution using Slices ------------------------

\ssection{IMPLEMENTATION}

The algorithm described here for visualizing smoke and fire realistically consists of placing
planes within the solution domain and determining opacity and color
at various locations within these planes
using data generated by a fire model such as FDS.  The video card then is
used when drawing smokeview to form a solution by combining colors and opacities in the currently drawn
plane with those already accumulated in a screen buffer.
The original Smokeview algorithm placed planes exactly where data was recorded by FDS,
either parallel to the XY, XZ or YZ axes planes or diagonally to two of these planes.
The algorithm described below places planes in more general locations. Interpolation is then used
within a 3D data set to obtain data values at locations required by the algorithm.
This increased flexibility
allows one to more easily reduce the amount of data accessed to draw smoke.
Planes are placed perpendicular to the line of sight and are equally spaced though not
necessarily at the same spacing as used by the underlying FDS grid.
This results in faster visualizations with the caveat that reduced data may result
in less realistic visualizations.  A brief overview of the algorithm is given below.

\begin{enumerate}
\item Given spacing parameters parallel and perpendicular to the line of sight, place
equally spaced planes
through the data, each plane oriented perpendicular to the line of sight.
This is illustrated in Figure~\ref{fig:smokeplanes}. Plane locations change as
the scene is moved.  By placing planes only where smoke is located, as illustrated in
Figure~\ref{fig:smokebox}, faster visualizations result.
If the graphics processing unit (GPU) is used to visualize smoke then data is passed to the GPU along with
vertices of the polygon of the intersected plane.  If the central processing unit (CPU) is used to visualize smoke then
the polygon needs to be triangulated (the GPU triangulates data within any triangle it draws).


\begin{figure}[bph]
\begin{center}
\includegraphics[height=4.0in]{../SMV_User_Guide/SCRIPT_FIGURES/smokegeom_fullbox}
\end{center}
\caption{Planes are placed within the entire solution domain so that they are equally spaced and are oriented perpendicular to the line of sight. In this example, the solution domain is rotated and the number of planes is reduced to make them more visible}
\label{fig:smokeplanes}
\end{figure}


\begin{figure}[bph]
\begin{center}
\begin{tabular}{cc}
\includegraphics[height=3.0in]{../SMV_User_Guide/SCRIPT_FIGURES/smokegeom_smoke}
\includegraphics[height=3.0in]{../SMV_User_Guide/SCRIPT_FIGURES/smokegeom_smokebox}
\end{tabular}
\end{center}
\caption{To improve visualization performance, especially at the beginning of the
simulation when a fire is typically small, planes are placed only where smoke and fire is located resulting in faster visualizations (since
data does not need to be obtained or drawn where it would not be visible)}
\label{fig:smokebox}
\end{figure}

\item Each data plane must be triangulated when smoke is drawn by the CPU.  The data plane,
represented as a polygon, is divided into a series of equally sized triangles
(except for those triangles that border the polygon edge). To do this,
    a 2D coordinate
    system is constructed for the planar polygon. As illustrated in Figure~\ref{fig:smoketriangulate},
    let $\vvec{u}$ be a vector corresponding to the longest
    polygonal side.  Let $\vvec{v}$ be a vector corresponding to the side clockwise
    (from the point of view of the observer) from $\vvec{u}$. Orthogonal unit vectors, $\vvec{s}$
    and $\vvec{t}$,
    in the plane of the polygon are formed using

      \begin{eqnarray*}
    \vvec{s}&=&\vvec{u}/||\vvec{u}||,\\
    \vvec{t}&=&(\vvec{s} \times \hat{\vvec{v}} ) \times \hat{\vvec{v}}
    \end{eqnarray*}
    where $\hat{\vvec{v}}=\vvec{v}/||\vvec{v}||$.

\begin{figure}[bph]
\begin{center}
\begin{tabular}{cc}
\includegraphics[height=3.5in]{../../../fig/smv/figures/smokegeom_5p5265poly}&
\includegraphics[height=3.5in]{../../../fig/smv/figures/smokegeom_triangulation}\\
a) intersection of a plane and a solution mesh& b) triangulated polygon
\end{tabular}
\end{center}
\caption{Intersection of a plane perpendicular to the line of sight and the solution
domain.  This results in a polygon which is triangulated
using a  2D coordinate system represented by vectors $\vvec{s}$ and $\vvec{t}$ located in the plane of this polygon.
Similar polygons uniformly spaced and perpendicular to the line of site are also generated and triangulated whenever the scene is moved}
\label{fig:smoketriangulate}
\end{figure}

\item For a spacing parameter $\Delta$, the polygon is then triangulated in terms of vectors $\vvec{s}$ and $\vvec{t}$ by first
forming vertices $v_{ij}=x_0 + i\vvec{s} + j \vvec{t}$ for $i$ and $j$ so that $v_{ij}$ covers the polygon (and the region just next to it)
and next forming triangles $(v_{ij},v_{i+1,j},v_{i+1,j+1}), (v_{ij},v_{i+1,j+1},v_{i,j+1})$
from these vertices.
Triangles with all vertices outside the polygon are neglected.
If one or two vertices are outside the polygon they are moved to the closest point on the polygon.

\end{enumerate}

The slice orientation
is chosen to be the one most perpendicular to the viewer's line of
sight.  The opacity at each vertex is computed using the
distance between adjacent planes and soot density
data computed by the fire model.  Opacities are adjusted if the distance
between planes is different than the spacing parameter used to compute the original opacity.
Opacity data is
computed and compressed using run length encoding as a
preprocessing step and decompressed as each frame is
displayed.

Summarizing, a slice rendering algorithm for visualizing smoke consists of
splitting the RTE across individual slice planes within a single
mesh.  Figure~\ref{fig:smokenum} shows an example illustrating a various number
of slice planes used to visualize smoke and fire.
The 3D computational domain is partitioned into a series of
2D slices.  The RTE is then solved on each slice.  Each slice
solution only accounts for conditions between adjacent slices.
The individual partially transparent slice solutions are then
combined using video hardware to form the final image.
Slices become more transparent as the come closer together. Problems
can then occur resulting in poor  visualizations because of numerical round off error.
In this case, different solution techniques are required such as volume rendering which integrates the entire RTE at once rather than one slice at a time.

\begin{figure}[bph]
\begin{center}
\begin{tabular}{cc}
\includegraphics[height=3.5in]{../SMV_User_Guide/SCRIPT_FIGURES/smokegeom_smoke_06}&
\includegraphics[height=3.5in]{../SMV_User_Guide/SCRIPT_FIGURES/smokegeom_smoke_09}\\
a) 6 planes&9 planes\\
\includegraphics[height=3.5in]{../SMV_User_Guide/SCRIPT_FIGURES/smokegeom_smoke_12}&
\includegraphics[height=3.5in]{../SMV_User_Guide/SCRIPT_FIGURES/smokegeom_smoke_15}\\
a) 12 planes&15 planes\\
\end{tabular}
\end{center}
\caption{Four images showing increasing number of planes used to visualize smoke and fire}
\label{fig:smokenum}
\end{figure}

% -------------------  Computing Opacity ------------------------

\ssubsection{Computing Opacity}

Consider a ray traveling from the background to the
observer through intervening smoke. Light is absorbed or scattered
by the smoke as the ray passes through each slice plane. Emission from the flame or hot smoke is
implemented by coloring the smoke.  Scattering is implemented  using the total mass
extinction coefficient.  Light losses are assumed to be from both
absorption and scattering. Obscuration or opacity is computed along each
ray one grid plane at a time, using the Beer-Lambert law as
follows.  The $\alpha=1-\tau$ values are pre-computed by FDS using
the Beer-Lambert law~\cite{Siegel:2001}.  The $\alpha$ parameter represents an
opacity, 0.0, for completely transparent, and 1.0 for completely
opaque and is given by

\begin{equation}
\label{eq:alpha}
\alpha=1-\exp(-\sigma_t\Delta x)
\end{equation}

\noindent for the view direction down the $x$~axis
where $\Delta x$ is the distance between two grid planes and
$\sigma_t$ is the total mass extinction
coefficient.
The $\alpha$ parameter in equation (\ref{eq:alpha}) is used by Smokeview
to blend the smoke plane currently being drawn with the
background. For a different plane spacing, $\Delta \hat{x}$,
the opacity is adjusted using $\hat{\alpha}=1-(1-\alpha)^{\Delta \hat{x}/\Delta x}$.

% -------------------  Computing Color ------------------------

\ssubsection{Computing Color}

Smokeview visualizes smoke and fire by drawing a series of triangles in equally spaced parallel planes.
Color for these triangles are assigned by mapping temperature or HRRPUV (heat release per unit volume)
values
to color such as with a color map illustrated in Figure~\ref{fig:colormaps}.
Transparency for these triangles is assigned using soot density, the greater the soot density,
the more opaque the triangle.

The example color map in Figure~\ref{fig:colormaps} is split into two parts.  The left half is used
for coloring non-burning regions, the right half is used for coloring burning regions.
An HRRPUV cutoff value denoted ${\rm hrrpuv}_{\rm cutoff}$ is used to
distinguish these two regions.
If an HRRPUV value is below the cutoff,
smoke is drawn using colors from the left half of the color map, while if an HRRPUV value is
greater than the cutoff,
fire is drawn using colors from the right half of the color map.  The color map is defined as a table
of 256 red, green, blue color triplets.  A formula giving a color index for a given HRRPUV value is given by

\newcommand{\hrr}{{\rm hrr}}
\newcommand{\hrrcutoff}{{\rm hrr}_{\rm cutoff}}
\newcommand{\hrrmax}{{\rm hrr}_{\rm max}}

\begin{eqnarray}
\mbox{color index}=\left\{
\begin{array}{ll}
  127\frac{\hrr}{\hrrcutoff} & 0 \le \hrr \le \hrrcutoff \\
  127 + 128\frac{\hrr-\hrrcutoff}{\hrrmax-\hrrcutoff} & \hrrcutoff \le \hrr \le \hrrmax
\end{array}
\right.
\end{eqnarray}

\begin{figure}[bph]
\begin{center}
\includegraphics[width=5.0in]{\SMVfigdir/colorbar_fire2}
\end{center}
\caption{Example colormap used for converting temperature or HRRPUV values to color}
\label{fig:colormaps}
\end{figure}

A second method for coloring fire is illustrated in Figure~\ref{fig:blendedmaximages}.
Instead of blending colors found along the line of sight it uses the color corresponding to the maximum temperature found along that path. This is implemented in Smokeview using a feature of the open graphics library
(OpenGL)\cite{OpenGLRed}
that replaces the color in the background only if the currently drawn color has a greater value (in terms of the red, green or blue color components).  Though the visualizations look more realistic it is not as flexible since the background has to be black (zero red, green blue color values) in order for the maximum replacement feature to work.

\begin{figure}[bph]
\begin{center}
\begin{tabular}{cc}
\includegraphics[width=3.6in]{\SMVfigdir/blended_hrrpuv_5s}&
\includegraphics[width=3.6in]{\SMVfigdir/max_temp_5s}
\end{tabular}
\end{center}
\caption{The image on the left was generated by blending colors found along each line of sight using opacities derived from smoke density.  The image on the right was generated by using the color corresponding
to the maximum temperature found along each line of sight}
\label{fig:blendedmaximages}
\end{figure}


% -------------------  Future Work ------------------------

\ssection{SUMMARY}

This \paper\ describes how Smokeview uses a simplified form of the radiative transport equation to display
smoke and fire.
This equation is solved using either the CPU or the GPU along with the video card by drawing a series of slices.
Each slice is triangulated and drawn using opacities derived from Beer's law a simplified form of the
radiative transport equation. The slices are colored using color maps. 
These algorithms
may be improved in several ways. The integration of the radiative transport equation can have problems
due to the limited numerical precision (8 bits) used to represent smoke opacity.
One solution to this as illustrated in Figure~\ref{fig:blendedmaximages} is to use
choose maximum values rather than blending values.  A better solution is to use data with
more precision to perform the integration. Slice coloring may be made more quantitative by
relating temperature to color using physics based rather than an assumed
color map. 

\bibliographystyle{unsrt}
\bibliography{../../../fds/Manuals/Bibliography/FDS_general,../../../fds/Manuals/Bibliography/FDS_refs,../../../fds/Manuals/Bibliography/FDS_mathcomp,../Bibliography/sv_fire,../Bibliography/sv_graphics}

\end{document}
