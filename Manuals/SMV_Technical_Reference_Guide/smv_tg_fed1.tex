\section{Computing and Visualizing Fractional Effective Dose data}
The fractional effective dose (FED), developed by
Purser~\cite{SFPE:Purser}, is a measure of human incapacitation
due to combustion gases.  FED index data is computed by Smokeview
using CO, $\mathrm{CO_2}$ and $\mathrm{O_2}$ gas concentration
data computed by FDS. Future work involves incorporating other
constituents such as soot or HCN.  Smokeview obtains this data
from FDS using slice files.

The total FED is computed here in terms of FED components for CO
and $\mathrm{O_2}$.  A hyper-ventilating factor due to
$\mathrm{CO_2}$ is applied to the FED for CO. The total FED is
then given by

\be \mathrm{FED}_\mathrm{tot} = \mathrm{FED}_\mathrm{CO} \times
\mathrm{HV}_\mathrm{CO_2} + \mathrm{FED}_\mathrm{O_2} \ee

Other terms involving CN, NOx and irritants are neglected. The
fraction of an incapacitating dose due to CO,
$\mathrm{FED}_\mathrm{CO}$ is calculated using

\be \mathrm{FED}_\mathrm{CO}(t) = \int_0^t 2.764 \times 10^{-5} \,
(C_\mathrm{CO}(t))^{1.036} \, dt \label{eq:fedCO} \ee

where $t$ is time in minutes and $C_\mathrm{CO}$ is the CO
concentration (ppm). The fraction of an incapacitating dose due to
low O${}_2$ hypoxia , $\mathrm{FED}_\mathrm{O_2}$, is calculated
using

\be \mathrm{FED}_\mathrm{O_2}(t) =  \int_0^t \frac{dt}{\exp \left
[ 8.13 - 0.54 \, (20.9 - C_\mathrm{O_2}(t)) \right ] }
\label{eq:fedO2} \ee

where $t$ is time in minutes and $C_\mathrm{O_2}$ is the O${}_2$
concentration (volume per cent). The hyperventilation factor
induced by carbon dioxide, $\mathrm{HV}_\mathrm{CO_2}$, is
calculated using

\be \mathrm{HV}_\mathrm{CO_2}(t) = \frac{ \exp( 0.1903 \,
C_\mathrm{CO_2}(t) +  2.0004 ) }{7.1} \label{eq:co2hyp} \ee

where $t$ is time in minutes and $C_\mathrm{CO_2}$ is the
$\mathrm{CO_2}$ concentration (percent).

\subsection{FED example}
Assuming that $C_\mathrm{CO}(t)=\mathrm{CO}$,
$C_\mathrm{CO_2}(t)=\mathrm{CO_2}$ and
$C_\mathrm{O_2}(t)=\mathrm{O_2}$ are constant, equations
\ref{eq:fedCO}, \ref{eq:fedO2} and \ref{eq:co2hyp} reduce to

\be \mathrm{FED}_\mathrm{CO}(t) = 2.764 \times 10^{-5} \,
\mathrm{CO}^{1.036} \, t \label{eq:fedCOcons} \ee

\be \mathrm{FED}_\mathrm{O_2}(t) =   \exp( -8.13 + 0.54 \, (20.9 -
\mathrm{O_2}) )t \label{eq:fedO2cons} \ee

\be
\mathrm{HV}_\mathrm{CO_2}(t) = \frac{ \exp( 0.1903 \, \mathrm{CO_2} +  2.0004 ) }{7.1}
\label{eq:fedCO2cons}
\ee

so that the total FED is given by

\begin{eqnarray}
\mathrm{FED}_\mathrm{tot}
&= &\mathrm{FED}_\mathrm{CO}(t)\times\mathrm{HV}_\mathrm{CO_2}(t)+\mathrm{FED}_\mathrm{O_2}(t)\\
\nonumber
 &= &\left(2.764 \times 10^{-5} \, \mathrm{CO}^{1.036}\times
\frac{ \exp( 0.1903 \, \mathrm{CO_2} +  2.0004 ) }{7.1} + \exp( -8.13
+ 0.54 \, (20.9 - \mathrm{O_2}) )\right) t\\
\end{eqnarray}

Figure \ref{fig:fedplot} presents two FED computations where CO,
$\mathrm{CO_2}$ and $\mathrm{O_2}$ are constant.
The first for a smoke filled room with CO=\SI{10000}{ppm},
$\mathrm{CO_2}=5 \%$ and $\mathrm{O_2}=10 \%$ and the second for a
room with ambient conditions with CO=\SI{0}{ppm},
$\mathrm{CO_2}=0.04 \%$ and $\mathrm{O_2}=21 \%$

\begin{figure}[bph]
\begin{center}
\includegraphics[width=5.0in]{\SMVfigdir/fed_clear}\\
a) ambient\\
\includegraphics[width=5.0in]{\SMVfigdir/fed_smoke}\\
b) smoke filled
\end{center}
\caption
[Plot of FED vs. time for ambient and smoke filled rooms]
{Plot of FED vs. time for ambient and smoke filled rooms.}
\label{fig:fedplot}%
\end{figure}
